\section{Motivation}
As a motivation example, consider Figure~\ref{Wuhan} which depicts two heterogeneous \KGs extracted from the \FIXME{xx} and \FIXME{xx}
datasets. In the diagram, an entity is marked by a circle while an attribute or relation is listed along the edge of a \KG.

As can be seen from the diagram, although the two \KGs have different graph structures, they contain an entity in common, \emph{Wuhan}.
Therefore, a successful entity alignment strategy would link the \emph{Wuhan} entity from both \KGs. However, the state-of-the art entity
alignment methods~\cite{} built upon TransE all fail to align this entity because of two reasons: (i) they are misled by some of the
redundant entities and attributes which do not appear on both \KGs and (ii) \FIXME{they do not utilize attributes like xx}. This is a
problem when using translation-based embeddings.

If we look closely into the \KGs, we find that the \emph{Wuhan} entity can be aligned using the two entities marked with \FIXME{circles
with dot lines} together with attribute \FIXME{XX}. If we can recognize and treat other entities and attributes as noise for this task, we
can then successfully align the entity by only considering the highlighted entities and attribute.

This example shows that entity alignment requires one to carefully evaluate the importance of each entity and attribute from heterogeneous
\KGs. Unfortunately, this information varies across datasets and entities. Because manually obtaining this information for every target
entity would incur significant overhead, there is a critical need to automate the process. In this paper, we describe an novel approach to
offer this capability based on the \GCN.
