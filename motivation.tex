\section{Motivation}
As a motivation example, consider Figure~\ref{Wuhan} which depicts two heterogeneous \KGs extracted from the \FIXME{xx} and \FIXME{xx}
datasets. In the diagram, an entity is marked by a circle while an attribute or relation is listed on the edge.

Although the two \KGs have different graph structures, they both contain a common entity of \emph{Wuhan}. Therefore, a successful entity
alignment strategy needs to link \emph{Wuhan} from both \KGs. However, the state-of-the art entity alignment methods~\cite{} built upon
TransE all fail to align this entity because of two reasons: (i) they are misled by some of the redundant entities and attributes that only
appear in one of \KGs and (ii) \FIXME{they do not utilize attributes like xx}. This is a problem when using translation-based embedding.

If we look closely into the \KGs, we find that the \emph{Wuhan} entity can be aligned using the two entities highlighted by \FIXME{dot-line
circles} together with attribute \FIXME{XX}. In this example, if we can recognize and treat other entities and attributes as noise, we can
then successfully align the \emph{Wuhan} entity.

This example shows that entity alignment requires one to carefully evaluate the importance of each entity and attribute from heterogeneous
\KGs. Unfortunately, this information varies across datasets and entities. Because manually obtaining this information for every target
entity would incur significant overhead, there is a critical need to automate the process. In this paper, we describe an novel approach to
offer this capability based on the \GCN.
