	\begin{abstract} Entity alignment is the task of establishing the link between entities across different knowledge graphs (\KGs),
	% - is an important task for natural language processing.
	which is recently performed on the \KGs by using graph embedding methods to represent the entities and their context. However, existing embedding-based approaches often fail to identify the sparse but discriminative evidence from heterogenous \KGs that usually have different
structures and properties. As a result, existing entity alignment methods have to either rely on heavily annotated training data which
are difficult to construct or compromise on the alignment accuracy. Our work introduces a better way for entity alignment. We employ the
relational graph convolutional network to better characterize entities by incorporating the neighboring relational structures and the
entity semantic and attribute information. Further, we show that by introducing layer-wise highway network gates to the model, our
approach can effectively filter out the propagation of noisy information to ensure the quality of the learning. Experiments performed on
two real-world datasets show that our approach outperforms prior methods but using much less annotated information.

		%Entity alignment is the task of linking entities and their counterparts among multiple knowledge graphs (\KGs).
%		Most recent works rely on knowledge graph embedding methods, such as TransE, to represent entities and their context, which often have difficulties in identifying sparse but discriminative evidence, utilizing attribute values and dealing with unbalanced neighboring context.
%		To address these issues, we present RGCN, a novel entity alignment model which leverages relational graph convolutional networks to better characterize entities by incorporating the neighboring relational structure information and taking attribute values into account.
%		We further introduce layer-wise highway network gates to filter out the propagation of noisy information and ensure effective spread of more informative and discriminative neighborhood information.
%		Experiments on real-world datasets show that our approach further improves entity alignment performance on various \KGs, and gets the best performance compared with competitive baselines.
	\end{abstract}
