	
	\section{Introduction}
	\label{section:intro}
    Acquiring knowledge from large volumes of unstructured textual data plays an important role in not only the development of Semantic
    Web, but also the understanding of massive text content. There has been an extensive body of work in transforming online encyclopedia resources, such as
    Wikipedia, into structured knowledge bases (\KBs) in a form of $\langle$\emph{subject entity},
   \emph{ predicate/relation}, \emph{object}$\rangle$ triples, such as DBpedia~\cite{Lehmann2009DBpedia,Auer2007DBpedia},
    Freebase~\cite{Bollacker2007Freebase}, Yago~\cite{Suchanek2008YAGO}, etc.
    	

   %In order to better show the relations among entities,

	Such kind of \KBs can be naturally organized into the form of graphs which we call knowledge graphs (\KGs). \KGs originating from the
same resource, e.g., Wikipedia, are usually created independently, thus often use different expressions and surface forms to indicate
equivalent entities and relations -- let alone those built from different resources, or even different languages. This common problem of
heterogeneity makes it extremely challenging to perform tasks like knowledge sharing, complementary and integration among different \KGs.
	
	One of the key enabling techniques to integrate different \KGs is \emph{entity alignment} -- the task of linking the equivalent
entities from different \KGs if they refer to the same real-world identity, usually with different surface forms. However, entity alignment
is non-trivial, which has to be realized through a complex alignment system~\cite{gokhale2014corleone,scharffe2014ontology}.
Traditional approaches generally rely on external information (such as hyperlinks in web pages). This limits the practicality of the
approach because it requires a labour-intensive and time-consuming process to construct an alignment model for each
\KG~\cite{zhu2017iterative}.


\begin{table}[t!]
	\centering
	\scriptsize
	\begin{tabular}{lllll}
		\toprule
		\bf Prior approach & \bf Entities & \bf Relations & \bf Triples & \bf Use of values \\
		\midrule
		\rowcolor{Gray} JE~\cite{hao2016joint} & ${\surd}$ & & & \\
		MTransE~\cite{chen2016multilingual} & $ $ & $ $ & ${\surd}$ & \\
		\rowcolor{Gray} ITransE~\cite{zhu2017iterative} & ${\surd}$ & ${\surd}$& & \\
		JAPE~\cite{sun2017cross} & ${\surd}$& ${\surd}$& & ${\surd}$ (type)\\
		\bottomrule
	\end{tabular}
	\caption{\small The use of seed alignment information in prior methods.}
	\label{seed}
\end{table}
	
	Most recently, efforts have been devoted to the so-called \KG embedding-based approaches. The idea is to use a translation-based model,
such as TransE~\cite{bordes2013translating}, to learn entity representations to jointly embed the structures of multiple \KGs into a
unified feature space where entity alignment can be easily performed. %To iron out the heterogeneity issue across different \KGs, such approaches rely
%on the pre-aligned entities and/or relations.

However, the current \KG embedding-based approaches have three significant drawbacks that limit the uptake of this promising technique and
the scale at which it can operate. Firstly, existing approaches cannot effectively handle situations where two entities' neighboring nodes
have significantly different relational structures - a common problem in heterogeneous \KGs. This is because current approaches cannot
identify which of the common entities and attributes are important for the alignment task (Section~\ref{sec:motivation}). Secondly, as we
summarize in Table~\ref{seed}, existing methods all rely on some forms of high-quality seed alignment data, such as pre-aligned \KG
predicates/relations or triples. Since those data structures in heterogeneous \KGs may be greatly different, it is very hard, if not
impossible, to collect such data in practice. Furthermore, most of the prior approaches do not consider the specific attribute values in
the \KGs because of the complex and diverse forms of attributes. As we will show later in the paper, the attribute values are often
important for entity alignment and cannot be ignored. JAPE~\cite{sun2017cross} was the first attempt to consider attributes for entity
alignment. However, it only considers the type but not the value of an attribute, thus misses massive opportunities
(Section~\ref{overall}).

Our work aims to address the aforementioned limitations of entity alignment. We do so by developing a new embedding-based framework based on the recently proposed relational graph convolutional network (\RGCN)~\cite{Schlichtkrull2017Modeling}. We demonstrate, for the first
time, that (i) how \RGCNs can be employed to better embed the highly multi-relational structure information of heterogeneous \KGs, (ii) how
an effective alignment model can be learned using only a small set of pre-aligned entities, and (iii) how highway network gates can be used
to control the accumulated noise across the network layers when processing large graphs. As a departure from prior work, our model exploits
both entity semantics and associated attribute values. It automatically learns which of the entities and attributes are important for each
specific alignment task, enabling the learned model to better handle heterogeneous graph structures where prior methods fail.

We evaluate our approach by applying it to two real-world datasets and testing it on over 50,000 entities with a large number of relations
and attributes. We compare our approach against state-of-the-art embedding-based
methods~\cite{hao2016joint,chen2016multilingual,sun2017cross,zhu2017iterative}. Experimental results show that our approach can
significantly outperform prior methods that need much more seed information to work on. Our work thus offers an effective but low-cost way
for performing entity alignment among heterogeneous \KGs.



	
	The technical contributions of this paper can be summarized as follows:
	\begin{itemize}
		\item We propose the first \RGCN-based entity alignment model for heterogeneous \KGs (Section~\ref {sec:model}).
		
        \item We show that the attribute values are important for aligning entities from heterogeneous \KGs and how this information can
            be exploited within an entity alignment framework (Section~\ref{sec:model}).

		\item Our work employs layer-wise highway gates in an \RGCN to control how much neighborhood information should be
passed (Section~\ref{section:hgcn}), which greatly improves the performance of an \RGCN when processing
large \KGs (Section~\ref{sec:results}).
		
		%\item We build a large-scale entity alignment dataset in Chinese, containing 57,240 entities, 3,563 relations, 28,595 attributes, 231,003 relation triples and 515,065 attribute triples. We manually aligned 16,969 entity pairs as the gold standards of entity alignment.
		
	\end{itemize}
